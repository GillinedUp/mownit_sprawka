\documentclass[12pt,a4paper]{article}
\usepackage[MeX]{polski}
\usepackage[utf8]{inputenc}
\usepackage{listings}
\usepackage{graphicx}
\usepackage{fancyvrb}
\usepackage{tabularx}
\usepackage{geometry}
\usepackage{multirow}
\usepackage{amsmath}
\geometry{
 a4paper,
 total={170mm,257mm},
 left=20mm,
 top=20mm,
}
\author{Yurii~Vyzhha}
\title{Metody Obliczeniowe w Nauce i Technice \\ Laboratorium 1 \\
  Arytmetyka Komputerowa \\ Sprawozdanie}
\begin{document}
  \maketitle
  \paragraph{Zadanie 2 Faktoryzacja LU}%\mbox{}\vspace{3mm}\\
  \begin{Verbatim}
    function [p, A] = LU(A)
        [n,~] = size(A);
        s = zeros(1,5);
        p = zeros(1,n);
        for i = 1:n
            s(i) = max(abs(A(i,:)));
            p(i) = i;
        end
        for k = 1:n-1
            rmax = 0;
            for i = k+1:n
                r = abs(A(i,k)/s(i));
                if r > rmax
                    rmax = r;
                    imax = i;
                end
            end
            if k ~= imax
                tmp = P(k);
                p(k) = p(imax);
                p(imax) = tmp;
                A([k imax],:) = A([imax k],:);
            end
            for i = k+1:n
                xmult = A(i,k)/A(k,k);
                A(i,k) = xmult;
                for j = k+1:n
                    A(i,j) = A(i,j) - xmult*A(k,j);
                end
            end
        end
    end
  \end{Verbatim}
  Funkcja działa w następujący sposób. Najpierw tworzymy dwa wektory: $s$ oraz $p$.
  $s$ jest wektorem skalowania takim, że
  $s_i = \underset{1 \leq j \leq n}{\mathrm{max}}\lvert A_{ij} \rvert$.
  $p$ jest wektorem reprezentującym macierz zamiany rzędów $P$. Tą macierz można otrzymać
  najpierw tworząc macierz zerową $n \times n$ oraz w miejscach tej macierzy $(i,p(i))$
  wpisać jedynki. W głównej pętli $1 \leq k \leq n-1$, gdzie $n$ to rozmiar macierzy A,
  szukamy elementu $rmax$ takiego, że
  $\mathrm{rmax} = \underset{k+1 \leq i \leq n}{\mathrm{max}} \lvert \frac{A_{ik}}{s_i} \rvert$.
  W danej interacji pętli, rmax jest pivotem, a imax jest indexem rzędu, w
  którym go znaleźliśmy. Jeśli $\mathrm{imax} \neq \mathrm{k}$ to zamieniamy
  rzędy miejscami, zapisując zmiany w vector $P$. Następnie w pętlin $i = k+1:n$ obliczamy
  współczynnik xmult, za pomocą którego redukujemy rzędy od $k+1$ do $n$. Ten współczynnik
  wpisujemy na miejśce $A_{ik}$; on będzie elementem macierzy $L$. Po wykonianiu się
  programu dostajemy na wyjściu wektor P oraz macierz zmienioną macierz $A$, w
  której są jednocześnie zapisane macierze $L$ oraz $U$. \newline
  Dla sprawdzenia poprawności danej funkcji, napisałem program, który sprawdza, czy
  dla $P$, $L$ i $U$, otrzymanych jako wyniki funkcji faktoryzującej, zachodzi
  $P*A=L*U$.
\begin{Verbatim}
    function a = CheckLU(A)
        [n,~] = size(A);
        [p, lu] = LU(A);
        P = zeros(n);
        for i = 1:n
            P(i,p(i)) = 1;
        end
        L = eye(n);
        U = zeros(n);
        for i = n:-1:1
            for j = n:-1:i
                U(n-j+1,n-i+1) = lu(n-j+1,n-i+1);
            end
        end
        for i = 2:n
            for j = 1:i-1
                L(i,j) = lu(i,j);
            end
        end
        X = P*A;
        Y = L*U;
        a = true;
        for i = 1:n
            for j = 1:n
                if abs(X(i,j)-Y(i,j)) > 1e4*eps(min(abs(X(i,j)),abs(Y(i,j))))
                    a = false;
                    break
                end
            end
        end
    end
\end{Verbatim}
W tej funkcji oblicamy $P$, $L$ i $U$ za pomocą wspomnianej wyżej funkcji oraz
tworzymy nowe macierze $X=P*A$ i $Y=L*U$. Dalej porównujemy każdy odpowiedni element
macierzy $X$ oraz $Y$ i zwracamy "prawdę", jeśli $X_{ij}$ jest prawie równe
$Y_{ij}$ dla każdego $1 \leq i \leq n$ oraz $1 \leq j \leq n$.


\end{document}
